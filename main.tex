\documentclass{article}
\usepackage[margin=1in]{geometry}
\usepackage[utf8]{inputenc}
\usepackage{hyperref}

\title{FY2019 Contribution Report}
\author{Asher Mancinelli \\ HPC/ML}

\begin{document}

\maketitle

\section{ExaSGD \\ 
  {\small Subproject of Exascale Computing Project}}

\subsection{Hiop}

Coordinating with team members at LLNL, I helped develop an in-house unit testing framework in preparation for porting \href{https://github.com/LLNL/hiop}{HiOp}'s core linear algebra kernels to run on GPU devices.
I also coordinated with the rest of the team to \textbf{develop our strategy} for porting kernels to GPU, and then \textbf{led a development team} to port HiOp's entire linear algebra library to run on GPU devices.
I also wrote the scripts to automate our testbed, running CI on machines at PNNL and ORNL to verify the kernels we ported to GPU.
The PI, Slaven Peles, said that without me "this project simply would not have been completed."
I am an author on a paper our team is currently producing, which will (hopefully) be published in a special edition of the Paralell Computing journal, outlining our methods on this project.

\subsection{ExaGO}

I was also tasked with setting up a testbed and CI pipeline for tests in ExaGO, a driver for using various nonlinear solvers (such as HiOp), which I completed.
We also ported several key kernels in this project to run on GPUs, again supported by our automated testing system.
I was nominated to give a talk on this project at NREL's SIAM CSE21 Minisymposium next calendar year, \textbf{contributing to the national field of grid optimization}.

\bigskip

\underline{I received an Outstanding Performance Award} for my work in the ECP, the \underline{highest internal PNNL award given}.
I was also asked to take the position of Software Development Thrust Lead for FY21 by the PI's of this project (Slaven Peles, Chris Oehmen, Lori Ross), in which I would \underline{lead developers from 5 national laboratories} in our software development efforts, as well as continue \textbf{contributing to our overall software strategy}.

\bigskip

In addition to my contributions to the above projects, I also \textbf{led and helped organize three hackathons}, in which up to \underline{55+ members of 5+ national laboratories} called in as we discussed strategies for porting particularly challenging kernels to GPUs.
As we discussed strategies, I would implement and test changes to our \verb|>100,000| line codebase in real time.
Although the technical knowledge of those on the call often far exceeded my own, I was able to contribute to the discussion, explain key components of the libraries, and live-code algorithms as members of the call made suggestions.

\section{PNNL HPC/ML Team}

I contributed and advised many projects through the HPC/ML team, though I will only mention select projects below:

\subsection{GatherV Development}

Under the guidance of Doug Baxter, I developed a novel implementation for the \verb|MPI_Gatherv| routine.
This implementation \textit{significantly} improves performance under certain workloads - better than OpenMPI, Intel, and MPICH's respective implementations.
We hope to publish our results after further testing.

\subsection{MLFlow Strategy Development}

Late this FY, Phil George reached out to me for guidance on PNNL's strategy for MLFlow, a tool for machine learning developers.
I designed a strategy to take this project in another direction which would:

\begin{itemize}
  \item \underline{Unencumber researchers} without HPC experience in their ML workflows,
  \item Lower barrier to entry to our compute capabilities for non-HPC-specialists,
  \item and increase awareness of Research Computing's compute capabilites.
\end{itemize}

This strategy better develops PNNL's ML capabilities, a high-priority area in which we would like to expand significantly.
I am overseeing two other developers on our team as they implement this strategy.

\subsection{Presentations}

I was asked to give several brown-bag talks on optimization, Python development, and inter-language development at PNNL's \textit{Python Users's Group}.
This \underline{contributed to PNNL's culture of computing}, as I helped several other projects understand how they might achieve massive performance boosts without significantly increasing development time.
I demonstrated how I can easily match the performance of the most popular and high-performing Python libraries currently available, and several other projects adopted these strategies as a result.
One such project was APBS, managed by Nathan Baker, who reached out for my expertise on the project after attending one of my brown-bags.

\bigskip

In addition to technical and leadership work, I made myself available to interns in the HPC/ML team to:

\begin{itemize}
  \item give technical guidance,
  \item help them network within the lab,
  \item bring them on to projects which allign with their interests and skills.
\end{itemize}

\section{APBS}

I made significant contributions to APBS, a protien-folding program with $>$30,000 users globally.
I rewrote and refactored significant portions of the code to unecumber domain scientists which would be otherwise unable to contribute.

\section{Boltzmann}

In this project, I worked directly with researchers from PNNL and UC Santa Barbara in simulating biological systems.
Having no prior knowledge of their domain, I rewrote core pieces of their program to achieve massive performance gains.
\textbf{I reduced the runtime of their simulation from \textit{two weeks} to \textit{eight minutes (a $>$2500x boost in performance)}}.
Our team's work was presented to a panel of external evaluators from academic institutions and was deemed \textit{"the posterchild for how PNNL should operate"}.
Because of my work and the success of the project, Tony Peurrung (Deputy Dir for S\&T) reached out to our team to learn more about our process to figure out why we were so successful.
Jim Ang and other leaders and chief scientists at PNNL were also informed of our success.
I received another Outstanding Performance Award for my contributions to this project.

\section{Conclusion}

\textbf{In my first six months on staff} at PNNL, my most significant contributions and accolades have been:

\begin{itemize}
  \item \underline{Two} Outstanding Performance Awards
  \item Lead and organized collaboration between 5+ natl. labs
  \item Developed long term lab-wide and nation-wide strategies
  \item Made technical contributions to projects that have had a lab-wide, nation-wide, and global impact
  \item Lead and developed several technical teams.
\end{itemize}

\bigskip

\end{document}
